%\documentclass[russian,english,10pt,a5paper,reqno]{amsart}
%\documentclass[russian,english,10pt,a4paper,reqno]{article}
%\documentclass[russian,english,11pt,b5paper,reqno,dviphfm]{amsbook}
%\documentclass[russian,english,10pt,a5paper,reqno,dviphfm]{amsbook}
%\documentclass[russian,english,10pt,a5paper,reqno,dviphfm]{amsbook}
\documentclass[russian,english,12pt,a4paper,reqno,dviphfm,oneside]{book}


%%\usepackage[pdftex,a5paper]{hyperref}
\usepackage[headings]{fullpage}
%\usepackage{fancyhdr}
%%\setlength{\textwidth}{114mm}
%%\setlength{\linewidth}{114mm}
%%\setlength{\textheight}{175mm}
% последние три команды позволяют настроить размеры текста на странице
%%\hoffset=-8mm
%%\voffset=-13mm
%% предшествующие две команды позволяют избавиться от отступов по-умолчанию
%%\setlength\paperheight{210mm}
%%\setlength\paperwidth{148mm}
%% Последние две команды пришлось написать в явном виде,
%% поскольку ничего другое pdflatex не понимал

\setlength{\footskip}{7mm}
\usepackage[T2A]{fontenc}
\usepackage[utf8]{inputenc}
\usepackage[russian]{babel}
\usepackage{amsmath}
\usepackage{amssymb}
\usepackage{amsfonts}
\usepackage{textcomp}
\usepackage[all]{xy}
\usepackage{amsthm}
\usepackage{graphicx}
%\usepackage[dvips]{graphicx}
\usepackage{wrapfig}
\usepackage{concrete}
\usepackage{eufrak}
%\usepackage{euler}
\usepackage{babelbib}
\usepackage{multirow}
\usepackage{multicol}
\usepackage{longtable}
\usepackage{cite}
\usepackage{ifthen}
\usepackage{array}
\usepackage{soul}
\usepackage{afterpage}
\usepackage{indentfirst}
\usepackage{varioref}
\usepackage[colorlinks = true,
            linkcolor = red,
            urlcolor  = blue,
            citecolor = blue,
            anchorcolor = blue]{hyperref}

%атрибуты
\begin{document}
	
\pdfoutput=1
\selectlanguage{russian}
\title{Некоторые ПДД}
\author{М.О. Проскурин}
\date{\today}
\maketitle

\chapter*{Введение}
\addcontentsline{toc}{chapter}{Введение}
Настоящие Правила дорожного движения устанавливают единый порядок дорожного движения на всей территории Российской Федерации. Другие нормативные акты, касающиеся дорожного движения, должны основываться на требованиях Правил и не противоречить им.\\
Т.к. описывать все правила дорожного движения очень долго и сложно, я добавил в эту книгу лишь их часть. Для ознакомления с полными и актуальными правилами обратитесь на \href{http://www.gibdd.ru/docs/pprf/314/}{официальный сайт ГИБДД}.
\vspace{110pt}
{\flushright \Large{«Тише едешь - дальше будешь.\\
Опасайся бед, пока их нет.\\
Гляди в оба, да не разбей лоба.\\
Ехал прямо, да попал в яму.» \\} 
\normalsize {\textit {Неизвестные автолюбители\\}}
}
\vfill{\small{{Картинки со знаками, разметкой и примерами я сюда добавлять, конечно же, не буду}}

\chapter{Некоторые понятия и термины}
%\flushleft

\section{Термины, начинающиеся с букв А -- Д}

\subsection{Автомагистраль} 
Дорога, обозначенная знаком 5.1 и имеющая для каждого направления движения проезжие части, отделенные друг от друга разделительной полосой (а при ее отсутствии - дорожным ограждением), без пересечений в одном уровне с другими дорогами, железнодорожными или трамвайными путями, пешеходными или велосипедными дорожками.

\subsection{Автопоезд}
Механическое транспортное средство, сцепленное с прицепом (прицепами).

\subsection{Велосипедист}
Лицо, управляющее велосипедом.

\subsection{Велосипедная дорожка}
Конструктивно отделенный от проезжей части и тротуара элемент дороги (либо отдельная дорога), предназначенный для движения велосипедистов и обозначенный знаком 4.4.1.

\subsection{Водитель}
Лицо, управляющее каким-либо транспортным средством, погонщик, ведущий по дороге вьючных, верховых животных или стадо. К водителю приравнивается обучающий вождению.

\subsection{Вынужденная остановка}
Прекращение движения транспортного средства из-за его технической неисправности или опасности, создаваемой перевозимым грузом, состоянием водителя (пассажира) или появлением препятствия на дороге.

\subsection{Главная дорога}
Дорога, обозначаемая знаками 2.1, 2.3.1 - 2.3.7 или 5.1, по отношению к пересекаемой (примыкающей), или дорога с твердым покрытием (асфальто- и цементобетон, каменные материалы и тому подобные) по отношению к грунтовой, либо любая дорога по отношению к выездам с прилегающих территорий. Наличие на второстепенной дороге непосредственно перед перекрестком участка с покрытием не делает ее равной по значению с пересекаемой.

\subsection{Дневные ходовые огни}
Внешние световые приборы, предназначенные для улучшения видимости движущегося транспортного средства спереди в светлое время суток.

\subsection{Дорога}
Обустроенная или приспособленная и используемая для движения транспортных средств полоса земли либо поверхность искусственного сооружения. Дорога включает в себя одну или несколько проезжих частей, а также трамвайные пути, тротуары, обочины и разделительные полосы при их наличии.

\subsection{Дорожное движение}
Совокупность общественных отношений, возникающих в процессе перемещения людей и грузов с помощью транспортных средств или без таковых в пределах дорог.

\subsection{Дорожно-транспортное происшествие}
Событие, возникшее в процессе движения по дороге транспортного средства и с его участием, при котором погибли или ранены люди, повреждены транспортные средства, сооружения, грузы либо причинен иной материальный ущерб.

\newpage
\section{Термины, начинающиеся с букв Ж -- Н}

\subsection{Железнодорожный переезд}
Пересечение дороги с железнодорожными путями на одном уровне.

\subsection{Маршрутное транспортное средство}
Транспортное средство общего пользования (автобус, троллейбус, трамвай), предназначенное для перевозки по дорогам людей и движущееся по установленному маршруту с обозначенными местами остановок.

\subsection{Механическое транспортное средство}
Ттранспортное средство, кроме мопеда, приводимое в движение двигателем. Термин распространяется также на любые тракторы и самоходные машины.

\subsection{Мопед}
Двух- или трехколесное транспортное средство, приводимое в движение двигателем с рабочим объемом не более 50 куб. см и имеющее максимальную конструктивную скорость не более 50 км/ч. К мопедам приравниваются велосипеды с подвесным двигателем, мокики и другие транспортные средства с аналогичными характеристиками.

\subsection{Мотоцикл}
Двухколесное механическое транспортное средство с боковым прицепом или без него. К мотоциклам приравниваются трех- и четырехколесные механические транспортные средства, имеющие массу в снаряженном состоянии не более 400 кг.

\subsection{Населенный пункт}
Застроенная территория, въезды на которую и выезды с которой обозначены знаками 5.23.1 - 5.26.

\subsection{Недостаточная видимость}

Видимость дороги менее 300 м в условиях тумана, дождя, снегопада и тому подобного, а также в сумерки.

\section{Остальные термины}
Остальные термины и более современные формулировки прошу искать на \href{http://www.gibdd.ru//}{официальном сайте ГИБДД}.

\chapter{Общие положения}

$1.$ Участники дорожного движения обязаны знать и соблюдать относящиеся к ним требования Правил, сигналов светофоров, знаков и разметки, а также выполнять распоряжения регулировщиков, действующих в пределах предоставленных им прав и регулирующих дорожное движение установленными сигналами.\\

$2.$ На дорогах установлено правостороннее движение транспортных средств.\\

$3.$ Участники дорожного движения должны действовать таким образом, чтобы не создавать опасности для движения и не причинять вреда.

Запрещается повреждать или загрязнять покрытие дорог, снимать, загораживать, повреждать, самовольно устанавливать дорожные знаки, светофоры и другие технические средства организации движения, оставлять на дороге предметы, создающие помехи для движения. (статья 12.33 КоАП) Лицо, создавшее помеху, обязано принять все возможные меры для ее устранения, а если это невозможно, то доступными средствами обеспечить информирование участников движения об опасности и сообщить в полицию.\\

$4.$ Лица, нарушившие Правила, несут ответственность в соответствии с действующим законодательством.

\chapter{Общие обязанности водителей}

\section[Водитель механического транспортного средства обязан]{Водитель механического транспортного средства обязан:}
\subsection[Иметь и предоставить]{Иметь при себе и по требованию сотрудников полиции передавать им для проверки:}
\noindent- водительское удостоверение или временное разрешение на право управления транспортным средством соответствующей категории;\\
- регистрационные документы на данное транспортное средство, а при наличии прицепа - и на прицеп;\\
- в установленных случаях разрешение на осуществление деятельности по перевозке пассажиров и багажа легковым такси, путевой лист, лицензионную карточку и документы на перевозимый груз, а при перевозке крупногабаритных, тяжеловесных и опасных грузов - документы, предусмотренные правилами перевозки этих грузов;\\
- страховой полис обязательного страхования гражданской ответственности владельца транспортного средства в случаях, когда обязанность по страхованию своей гражданской ответственности установлена федеральным законом.\\
В случаях, прямо предусмотренных действующим законодательством, иметь и передавать для проверки работникам Федеральной службы по надзору в сфере транспорта лицензионную карточку, путевой лист и товарно-транспортные документы.

\subsection{Пристегнуться} При движении на транспортном средстве, оборудованном ремнями безопасности, быть пристегнутым и не перевозить пассажиров, не пристегнутых ремнями. При управлении мотоциклом быть в застегнутом мотошлеме и не перевозить пассажиров без застегнутого мотошлема.\\

\section[Международное движение]{Водитель механического транспортного средства, участвующий в международном дорожном движении, обязан:}
\noindent- иметь при себе регистрационные документы на данное транспортное средство (при наличии прицепа - и на прицеп) и водительское удостоверение, соответствующие Конвенции о дорожном движении;\\
- иметь на данном транспортном средстве (при наличии прицепа - и на прицепе) регистрационные и отличительные знаки государства, в котором оно зарегистрировано.\\
Водитель, осуществляющий международную автомобильную перевозку, обязан останавливаться по требованию работников Федеральной службы по надзору в сфере транспорта в специально обозначенных дорожным знаком 7.14 контрольных пунктах и предъявлять для проверки транспортное средство, а также разрешения и другие документы, предусмотренные международными договорами Российской Федерации. Отличительные знаки государства могут помещаться на регистрационных знаках.\\


\section[Другие обязательства]{Водитель транспортного средства также обязан:}

\subsection[Исправность ТС]{Держать свое ТС в исправности} Перед выездом проверить и в пути обеспечить исправное техническое состояние транспортного средства в соответствии с Основными положениями по допуску транспортных средств к эксплуатации и обязанностями должностных лиц по обеспечению безопасности дорожного движения.

Запрещается движение при неисправности рабочей тормозной системы, рулевого управления, сцепного устройства (в составе автопоезда), не горящих (отсутствующих) фарах и задних габаритных огнях в темное время суток или в условиях недостаточной видимости, недействующем со стороны водителя стеклоочистителе во время дождя или снегопада.

При возникновении в пути прочих неисправностей, с которыми приложением к Основным положениям запрещена эксплуатация транспортных средств, водитель должен устранить их, а если это невозможно, то он может следовать к месту стоянки или ремонта с соблюдением необходимых мер предосторожности;

\subsection[Освидетельствоание]{Проходить освидетельствование} По требованию должностных лиц, которым предоставлено право государственного надзора и контроля за безопасностью дорожного движения и эксплуатации ТС проходить освидетельствование на состояние алкогольного опьянения и медицинское освидетельствование на состояние опьянения. Водитель транспортного средства Вооруженных Сил Российской Федерации, внутренних войск Министерства внутренних дел Российской Федерации, инженерно-технических и дорожно-строительных воинских формирований при федеральных органах исполнительной власти, спасательных воинских формирований Министерства Российской Федерации по делам гражданской обороны, чрезвычайным ситуациям и ликвидации последствий стихийных бедствий обязан проходить освидетельствование на состояние алкогольного опьянения и медицинское освидетельствование на состояние опьянения также по требованию должностных лиц военной автомобильной инспекции.

В установленных случаях проходить проверку знаний Правил и навыков вождения, а также медицинское освидетельствование для подтверждения способности к управлению транспортными средствами.

\subsection [Предоставлять транспортное средство]{Предоставлять транспортное средство:}
\noindent- сотрудникам полиции, федеральных органов государственной охраны и органов федеральной службы безопасности в случаях, предусмотренных законодательством;\\
- медицинским и фармацевтическим работникам для перевозки граждан в ближайшее лечебно-профилактическое учреждение в случаях, угрожающих их жизни.\\
\indentПо требованию владельцев транспортных средств федеральные органы государственной охраны и органы федеральной службы безопасности возмещают им в установленном порядке причиненные убытки, расходы либо ущерб в соответствии с законодательством.\\

\section{Остановка регулировщиком} Право остановки транспортных средств предоставлено регулировщикам, а грузовых автомобилей и автобусов, осуществляющих международные автомобильные перевозки, в специально обозначенных дорожным знаком 7.14 контрольных пунктах - также работникам Федеральной службы по надзору в сфере транспорта.

Работники Федеральной службы по надзору в сфере транспорта должны быть в форменной одежде и использовать для остановки диск с красным сигналом либо со световозвращателем. Они могут пользоваться для привлечения внимания водителей дополнительным сигналом-свистком.

Лица, обладающие правом остановки транспортного средства, обязаны предъявлять по требованию водителя служебное удостоверение.\\

\section{Обязанности при ДТП} При дорожно-транспортном происшествии водитель, причастный к нему, обязан немедленно остановить (не трогать с места) транспортное средство, включить
аварийную сигнализацию и выставить знак аварийной остановки в соответствии с требованиями пункта 7.2 Правил, не перемещать предметы, имеющие отношение к происшествию.\\

\subsection{Гибель или ранение людей при ДТП} Если в результате дорожно-транспортного происшествия погибли или ранены люди, водитель, причастный к нему, обязан:
- принять меры для оказания первой помощи пострадавшим, вызвать скорую медицинскую помощь и полицию;\\
- в экстренных случаях отправить пострадавших на попутном, а если этоневозможно, доставить на своем транспортном средстве в ближайшую медицинскую организацию, сообщить свою фамилию, регистрационный знак транспортного средства (с предъявлением документа, удостоверяющего личность, или водительского удостоверения и регистрационного документа на транспортное средство) и возвратиться к месту происшествия;\\
- освободить проезжую часть, если движение других транспортных средств невозможно, предварительно зафиксировав, в том числе средствами фотосъемки или видеозаписи, положение транспортных средств по отношению друг к другу и объектам дорожной инфраструктуры, следы и предметы, относящиеся к происшествию, и принять все возможные меры к их сохранению и организации объезда места происшествия;\\
- записать фамилии и адреса очевидцев и ожидать прибытия сотрудников полиции.

\subsection{Вред имуществу при ДТП} Если в результате дорожно-транспортного происшествия вред причинен только имуществу, водитель, причастный к нему, обязан освободить проезжую часть, если движению других транспортных средств создается препятствие, предварительно зафиксировав, в том числе средствами фотосъемки или видеозаписи, положение транспортных средств по отношению друг к другу и объектам дорожной инфраструктуры, следы и предметы, относящиеся к происшествию, повреждения транспортных средств.

Если обстоятельства причинения вреда в связи с повреждением имущества в результате дорожно-транспортного происшествия или характер и перечень видимых повреждений транспортных средств вызывают разногласия участников дорожно-транспортного происшествия, водитель, причастный к нему, обязан записать фамилии и адреса очевидцев и сообщить о случившемся в полицию для получения указаний сотрудника полиции о месте оформления дорожно-транспортного происшествия. В случае получения указаний сотрудника полиции об оформлении документов о дорожно-транспортном происшествии с участием уполномоченных на то сотрудников полиции на ближайшем посту дорожно-патрульной службы или в подразделении полиции водители оставляют место дорожно-транспортного происшествия, предварительно зафиксировав, в том числе средствами фотосъемки или видеозаписи, положение транспортных средств по отношению друг к другу и объектам дорожной инфраструктуры, следы и предметы, относящиеся к происшествию, повреждения транспортных средств.

Если обстоятельства причинения вреда в связи с повреждением имущества в результате дорожно-транспортного происшествия, характер и перечень видимых повреждений транспортных средств не вызывают разногласий участников дорожно-транспортного происшествия, водители, причастные к нему, не обязаны сообщать о случившемся в полицию. В этом случае они могут оставить место дорожно-транспортного происшествия и:\\
- оформить документы о дорожно-транспортном происшествии с участием уполномоченных на то сотрудников полиции на ближайшем посту дорожно-патрульной службы или в подразделении полиции, предварительно зафиксировав, в том числе средствами фотосъемки или видеозаписи, положение транспортных средств по отношению друг к другу и объектам дорожной инфраструктуры, следы и предметы, относящиеся к происшествию, повреждения транспортных средств;\\ 
- оформить документы о дорожно-транспортном происшествии без участия уполномоченных на то сотрудников полиции, заполнив бланк извещения о дорожно-транспортном происшествии в соответствии с правилами обязательного страхования,\\ - если в дорожно-транспортном происшествии участвуют 2 транспортных средства (включая транспортные средства с прицепами к ним), гражданская ответственность владельцев которых застрахована в соответствии с законодательством об обязательном страховании гражданской ответственности владельцев транспортных средств, вред причинен только этим транспортным средствам и обстоятельства причинения вреда в связи с повреждением этих транспортных средств в результате дорожно-транспортного происшествия не вызывают разногласий участников дорожно-транспортного происшествия;\\
- не оформлять документы о дорожно-транспортном происшествии - если в дорожно-транспортном происшествии повреждены транспортные средства или иное имущество только участников дорожно-транспортного происшествия и у каждого из этих участников отсутствует необходимость в оформлении указанных документов.\\

\section[Водителю запрещается]{Водителю запрещается:}
\noindent- управлять транспортным средством в состоянии опьянения (алкогольного, наркотического или иного), под воздействием лекарственных препаратов, ухудшающих реакцию и внимание, в болезненном или утомленном состоянии, ставящем под угрозу безопасность движения;\\
- передавать управление транспортным средством лицам, находящимся в состоянии опьянения, под воздействием лекарственных препаратов, в болезненном или утомленном состоянии, а также лицам, не имеющим при себе водительского удостоверения на право управления транспортным средством данной категории или в случае его изъятия в установленном порядке - временного разрешения, кроме случаев обучения вождению в соответствии с разделом 21 Правил;
- пересекать организованные (в том числе и пешие) колонны и занимать место в них;\\
- употреблять алкогольные напитки, наркотические, психотропные или иные одурманивающие вещества после дорожно-транспортного происшествия, к которому он причастен, либо после того, как транспортное средство было остановлено по требованию сотрудника полиции, до проведения освидетельствования с целью установления состояния опьянения или до принятия решения об освобождении от проведения такого освидетельствования;\\
- управлять транспортным средством с нарушением режима труда и отдыха, установленного уполномоченным федеральным органом исполнительной власти, а при осуществлении международных автомобильных перевозок — международными договорами Российской Федерации;\\
- пользоваться во время движения телефоном, не оборудованным техническим устройством, позволяющим вести переговоры без использования рук.

\chapter{Скорость движения}

\section{Общие ограничения} Водитель должен вести транспортное средство со скоростью, не превышающей установленного ограничения, учитывая при этом интенсивность движения, особенности и состояние транспортного средства и груза, дорожные и метеорологические условия, в частности видимость в направлении движения. Скорость должна обеспечивать водителю возможность постоянного контроля за движением транспортного средства для выполнения требований Правил.

При возникновении опасности для движения, которую водитель в состоянии обнаружить, он должен принять возможные меры к снижению скорости вплоть до остановки транспортного средства.

\section{Ограничения в населенных пунктах} В населенных пунктах разрешается движение транспортных средств со скоростью не более 60 км/ч, а в жилых зонах и на дворовых территориях не более 20 км/ч.

\textit{Примечание:}

По решению органов исполнительной власти субъектов Российской Федерации может разрешаться повышение скорости (с установкой соответствующих знаков) на участках дорог или полосах движения для отдельных видов транспортных средств, если дорожные условия обеспечивают безопасное движение с большей скоростью. В этом случае величина разрешенной скорости не должна превышать значения, установленные для соответствующих видов транспортных средств на автомагистралях.

\section{Ограничения вне населенных пунктов}
Вне населенных пунктов разрешается движение:
- легковым автомобилям и грузовым автомобилям с разрешенной максимальной массой не более 3,5 т на автомагистралях — со скоростью не более 110 км/ч, на остальных дорогах — не более 90 км/ч;\\
- междугородним и маломестным автобусам и мотоциклам на всех дорогах — не более 90 км/ч;\\
- другим автобусам, легковым автомобилям при буксировке прицепа, грузовым автомобилям с разрешенной максимальной массой более 3,5 т на автомагистралях — не более 90 км/ч, на остальных дорогах — не более 70 км/ч;\\
- грузовым автомобилям, перевозящим людей в кузове, — не более 60 км/ч;\\
- транспортным средствам, осуществляющим организованные перевозки групп детей, — не более 60 км/ч.

\textit{Примечание:}

По решению собственников или владельцев автомобильных дорог может разрешаться повышение скорости на участках дорог для отдельных видов транспортных средств, если дорожные условия обеспечивают безопасное движение с большей скоростью. В этом случае величина разрешенной скорости не должна превышать значения 130 км/ч на дорогах, обозначенных знаком 5.1, и 110 км/ч на дорогах, обозначенных знаком 5.3.

\section{Ограничения при использовании буксира} Транспортным средствам, буксирующим механические транспортные средства, разрешается движение со скоростью не более 50 км/ч.

Транспортным средствам, перевозящим крупногабаритные, тяжеловесные и опасные грузы, разрешается движение со скоростью, не превышающей скорости, установленной при согласовании условий перевозки.

\section[Водителю запрещается]{Водителю запрещается:}
- превышать максимальную скорость, определенную технической характеристикой транспортного средства;\\
- превышать скорость, указанную на опознавательном знаке “Ограничение скорости”, установленном на транспортном средстве;\\
- создавать помехи другим транспортным средствам, двигаясь без необходимости со слишком малой скоростью;\\
- резко тормозить, если это не требуется для предотвращения дорожно-транспортного происшествия.\\



\chapter*{Заключение}
\addcontentsline{toc}{chapter}{Заключение}


Боюсь, что этод свод правил оказался немного более объемными, чем я считал изначально, поэтому тут я собрал всего лишь несколько глав. Возможно, если прочитать приложения, то можно узнать о ПДД немного больше. 

Спасибо за внимание!

\tableofcontents

\appendix
\chapter[Известные водители]{Известные водители}
\section{Михаэль Шумахер}
Михаэль Шумахер --- немецкий автогонщик Формулы-1. Семикратный чемпион мира, двукратный вице-чемпион мира и трижды бронзовый призёр. 47 лет. После несчастного случая на горных лыжах лежит в коме (по некоторым данным его состояние улучшается).
\section{Владимир Гостюхин}
Владимир Васильевич Гостюхин --- советский и белорусский актёр театра и кино, кинорежиссёр. Заслуженный артист Белорусской ССР. Народный артист Беларуси. Лауреат Государственной премии СССР и Государственной премии РФ.\\
Наиболее известен общественности, как Фёдор Иванович Афанасьев, дальнобойщик, из одноименного сериала. Коронная фраза:''Ёкарный бабай! Жми!''
\section{Владимир Путин}
Владимир Владимирович Путин --- российский государственный деятель, действующий президент Российской Федерации с 7 мая 2012 года. С 2000 по 2008 год -- второй президент Российской Федерации, с 2008 по 2012 год -- Председатель Правительства Российской Федерации. Также известен, как один из первых тестировщиков следующих автомобилей: Lada Kalina, Lada Granta, Lada Vesta.

\chapter{Интересные факты}
\section{Левостороннее и правостороннее движение}
Около $34\%$ всех водителей ездят по левой стороне. Такой немаленький процент достигается за счет жителей Индии, Пакистана, стран юга Африки и других бывших колоний Великобритании. А, ну и еще японцы, да.

\section{Первое ДТП}
Первое в истории зарегистрированное ДТП с участием автомобиля произошло 30 мая 1896 года в Нью-Йорке: электромобиль Генри Уэлса столкнулся с велосипедом Эвелина Томаса, который отделался переломом ноги.

\chapter{Полезные ссылки}
1) Решил еще раз добавить ссылку на \href{http://www.gibdd.ru/docs/pprf/314/}{сайт ГИБДД}.

2) Вот также \href{https://www.youtube.com/}{ссылка на youtube}, где можно посмотреть много увлекательных видео с автомоибльными авариями и сделать опрелеленные выводы.

3) Вот ещё \href{https://www.sharelatex.com/project/57ed251abbaba18a464a13c0}{ссылка} на \LaTeXe исходник этого документа.
\end{document}
%-- здесь наш документ закончился
\bye
%-- а чтобы ЛаТеХ не приставал с вопросами, мы с ним попрощались.
